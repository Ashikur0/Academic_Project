%%%%%%%%%%%%%%%%%%%%%%%%%%%%%%%%%%%%%%%%%
% Academic Title Page
% LaTeX Template
% Version 2.0 (17/7/17)
%
% This template was downloaded from:
% http://www.LaTeXTemplates.com
%
% Original author:
% WikiBooks (LaTeX - Title Creation) with modifications by:
% Vel (vel@latextemplates.com)
%
% License:
% CC BY-NC-SA 3.0 (http://creativecommons.org/licenses/by-nc-sa/3.0/)
% 
% Instructions for using this template:
% This title page is capable of being compiled as is. This is not useful for 
% including it in another document. To do this, you have two options: 
%
% 1) Copy/paste everything between \begin{document} and \end{document} 
% starting at \begin{titlepage} and paste this into another LaTeX file where you 
% want your title page.
% OR
% 2) Remove everything outside the \begin{titlepage} and \end{titlepage}, rename
% this file and move it to the same directory as the LaTeX file you wish to add it to. 
% Then add \input{./<new filename>.tex} to your LaTeX file where you want your
% title page.
%
%%%%%%%%%%%%%%%%%%%%%%%%%%%%%%%%%%%%%%%%%

%----------------------------------------------------------------------------------------
%	PACKAGES AND OTHER DOCUMENT CONFIGURATIONS
%----------------------------------------------------------------------------------------

\documentclass[11pt]{article}

\usepackage[utf8]{inputenc} % Required for inputting international characters
\usepackage[T1]{fontenc} % Output font encoding for international characters

\usepackage{mathpazo} % Palatino font
\usepackage{graphicx}
\usepackage{varioref}
\usepackage{float}

\begin{document}

%----------------------------------------------------------------------------------------
%	TITLE PAGE
%----------------------------------------------------------------------------------------

\begin{titlepage} % Suppresses displaying the page number on the title page and the subsequent page counts as page 1
	\newcommand{\HRule}{\rule{\linewidth}{0.5mm}} % Defines a new command for horizontal lines, change thickness here
	
	\center % Centre everything on the page
	
	%------------------------------------------------
	%	Headings
	%------------------------------------------------
	
	\textsc{\LARGE Institution Name}\\[1.5cm] % Main heading such as the name of your university/college
	
	\textsc{\Large LAB PROJECT}\\[0.5cm] % Major heading such as project name
	
	\textsc{\large COURSE CODE AND NAME}\\[0.5cm] % Minor heading such as course title
	
	%------------------------------------------------
	%	Title
	%------------------------------------------------
	
	\HRule\\[0.4cm]
	
	{\huge\bfseries PROJECT NAME}\\[0.4cm] % Title of your document
	
	\HRule\\[1.5cm]
	
	%------------------------------------------------
	%	Author(s)
	%------------------------------------------------
	
	\begin{minipage}{0.4\textwidth}
		\begin{flushleft}
			\large
			\textit{Group Member 1 Name}\\
			\textsc{Student ID} % Your name
		\end{flushleft}
	\end{minipage}
    ~
	\begin{minipage}{0.4\textwidth}
		\begin{flushright}
			\large
			\textit{Group Member 2 Name}\\
			\textsc{Student ID} 
		\end{flushright}
	\end{minipage}
	
	%------------------------------------------------
	%	Date
	%------------------------------------------------
	
	\vfill\vfill\vfill % Position the date 3/4 down the remaining page
	
	{\large\today} % Date, change the \today to a set date if you want to be precise
	
	%------------------------------------------------
	%	Logo
	%------------------------------------------------
	
	%\vfill\vfill
	%\includegraphics[width=0.2\textwidth]{placeholder.jpg}\\[1cm] % Include a department/university logo - this will require the graphicx package
	 
	%----------------------------------------------------------------------------------------
	
	\vfill % Push the date up 1/4 of the remaining page
	
\end{titlepage}

%----------------------------------------------------------------------------------------

\section{Description}
Your Project Description
\clearpage

\section{Motivation}
What motivated you to do this project?
\clearpage

\section{Novelty}

What is new about this project?

\begin{itemize}
\item Novel Feature 1
\item Novel Feature 2
\item Novel Feature 3
\end{itemize}

\clearpage

\section{Schema Diagram}
Schema Diagram of your project \\ 
\\
\includegraphics[scale=0.5]{Schema}

\clearpage

\section{Tools Used}

\begin{table}[H]
\centering
\caption{Tools used for the project}
\label{my-label}
\begin{tabular}{|l|l|}
\hline
\textbf{Tool} & \textbf{Purpose}         \\ \hline
Xampp         & For creating a localhost \\ \hline
phpMyAdmin    & For hosting the database \\ \hline
\end{tabular}
\end{table}

\clearpage

\section{Challenges Faced}
These are the challenges we faced while doing the project:
\begin{itemize}
\item \textbf{Challenge 1: } Write the first challenge.
\item \textbf{Challenge 2: } Write the second challenge.
\end{itemize}

\clearpage

\section{Applications}

Applications of our project is as follows:
\begin{itemize}
\item \textbf{Application 1: } Write the first application.
\item \textbf{Application 2: } Write the second application.
\end{itemize}

\clearpage

\section{Future Work}


\clearpage

\end{document}

